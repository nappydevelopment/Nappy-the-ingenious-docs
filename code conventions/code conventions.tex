\documentclass[fontsize=11,DIV=10]{scrartcl}
\KOMAoptions{footnotes=multiple}

\usepackage{hyperref}
\hypersetup{
	colorlinks=true,
	allcolors=black,
	linktoc=all,
	pdfauthor=Author,
	pdfencoding=auto
}

\usepackage[svgnames,table]{xcolor}
\usepackage{graphicx}

\usepackage{listings}

\usepackage{tocloft}
\renewcommand{\cftsecleader}{\cftdotfill{\cftdotsep}}

\usepackage{fontspec}
\usepackage{microtype}
%\setmainfont{DejaVu Serif} 	% Set fonts here
%\setsansfont{DejaVu Serif}
%\setmonofont{DejaVu Sans Mono}

%\usepackage{metalogo} % for \latex and such
%\usepackage{amsmath}

\usepackage{booktabs}	% better tables with \toprule \midrule and \bottomrule

\usepackage{enumitem}
\setlist[description]{style=nextline,labelsep*=3em}

\usepackage[yyyymmdd]{datetime}
\renewcommand{\dateseparator}{.}

% germanify stuff
\renewcommand{\contentsname}{Inhaltsverzeichnis}

\renewcommand{\listfigurename}{Abbildungsverzeichnis}
\renewcommand{\figurename}{Abbildung}

\renewcommand{\listtablename}{Tabellenverzeichnis}
\renewcommand{\tablename}{Tabelle}


\setlength{\parindent}{0pt}			% kein einzug
\setlength{\parskip}{0pt}			% kein vertical space zwischen pars
%\setlength{\mathindent}{0pt}		% kein mathe einzug


\begin{document}

{\Huge \bfseries Code Conventions v0.3 \hfill {\normalsize \today{}} \vspace{5pt}\\}
	This document is still under review and will change later on.

\vspace{2em}

\microtypesetup{protrusion=false}
\tableofcontents{}	\addtocontents{toc}{~\hfill\textbf{Seite}\par}
\microtypesetup{protrusion=true}

\newpage


\section{Tabs}
	This is one of points where teams often fight about.
	We will use a tab size of 4 spaces.


\section{Line length}
	Lines should not get longer than 140 characters (TODO: verify this)
	\subsection{Breaking function parameters}
		if they get too long, break like this:
		\begin{lstlisting}[tabsize=2]
			private static synchronized horkingLongMethodName(
					int anArg,
					Object anotherArg,
					String yetAnotherArg,
					Object andStillAnother
				){
					code();
			}
		\end{lstlisting}
	\subsection{Breaking if's}
		\begin{lstlisting}[tabsize=2]
			if( (condition1 && condition2)
					|| (condition3 && condition4)
					||!(condition5 && condition6)
				){
					doSomethingAboutIt();
			}
		\end{lstlisting}


\section{Getters and Setters}
	All variables are private!

	Getters and setters will be used for accessing and setting the variables, if needed.


\section{Function Naming}
	Marc will assign names to Functions.


\section{Variable Naming}
	\begin{description}
		\item[camelCase]{
			Variables start small and then every new Word is written big.
		}
		\item[Hungarian Notation]{
			We will use specifiers defining the Type of the variables.
			These will be mostly be one letter long and stand in front of the name.\\
			If a type of Object is used only once or twice it has the specifier o, else it got to have something else.
			This will help differentiating the Object types.
			% https://en.wikipedia.org/wiki/Hungarian_notation
		}
		\item[Constants]{
			written in big LETTERS and underscores split the words.
		}
	\end{description}
	\begin{tabular}{r l}
		\toprule
		type & specifier \\ \midrule
		integer & i \\
		unsigned integer & ui \\
		boolean & b \\
		string & s \\
		float & f \\
		double & d \\
		object\footnotemark{} & o \\
		array & a<data type> \\

		Character & chr \\

		textfield & txt \\
		textarea & txa \\
		Button & btn \\
		Menu & mnu \\
		label & lbl \\ \bottomrule
	\end{tabular}
	\footnotetext{If only one instance is used}
	\subsection{Example}
		iLoop for an integer which is used in loops \\
		sName for an String which holds a name \\



\section{Spacing}
	(TODO: discuss about this)
	\subsection{Space after if, for and while}
		no space after if, for or while and the opening parentheses.
	\subsection{Space between ) and \{}
		no space or newline between ) and \{.
	\subsection{Space in for loop definitions}
		\begin{lstlisting}[tabsize=2]
			for(int i = 2; i<1; i--){
				...
		\end{lstlisting}
	\subsection{One Statement in one Line}
		After each Statement insert one newline and tabs.
	\subsection{K\&R Style tabs and parentheses}
		K\&R Style tabs and parentheses:
		\begin{lstlisting}[tabsize=2]
			if(x < 0){
				System.out.println("Negative");
				negative(x);
			}else{
				System.out.println("Non-negative");
				nonnegative(x);
			}
		\end{lstlisting}


\section{Function Parameters}
		(TODO: discuss about this)
	\subsection{Ordering}
		From the most common variables to the ones that are changing most often.
	\subsection{All parameters are constant}
		The Parameters given to a function should be constant (final).

		The return statement is used for retuning stuff!


\section{Errors}
	If the function has enough knowledge to prevent errors, it should do so!

	If not then it should throw the Error to the caller.


	This should escalate till some function handles the Error.
	If the Error can not be handled then the Program should print the Error and exit (System.exit(1)).

	(TODO: use Common function for error printing and exit)


\section{No ternary expressions}
	This one should be simple.


\section{Fully Qualify Imports}
	no stars in imports!



\end{document}